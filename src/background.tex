\chapter{SPIN-UTI e SIMIOR}
\section{Implementazione di SPIN-UTI}
Il sistema SPIN-UTI nel reparto di terapia intensiva del Presidio Ospedaliero Duilio Casula è implementato a livello informatico con l'utilizzo dei software Microsoft Excel e Access, il primo utilizzato per inserire le informazioni e per effettuare i calcoli tramite formule, il secondo utilizzato come base dati.
Nel mondo informatico, l'utilizzo accoppiato di questi due software è noto per essere un sistema debole di organizzazione e salvataggio dati, dettato dai limiti di archiviazione dello stesso Access, alla rappresentazione più criptica dei dati in Excel (soprattutto se confrontati con un'implementazione ad-hoc di un sistema di visualizzazione).
Viene dunque logico, intuire l'inidoneità del sistema attuale, soprattutto sul fronte organizzativo dei dati.
\section{Limitazioni del sistema attuale}
La principale limitazione del sistema attualmente in esecuzione, è il limite di informazioni (dette \textit{record}) inseribili per ogni paziente, con la conseguente duplicazione delle schede contenenti tutte le informazioni sulla la degenza. 
Ne risulta un sistema poco ordinato e maggiormente soggetto a errori durante la copia delle informazioni essenziali.
Anche la raccolta delle statistiche (fulcro del sistema SPIN-UTI) è pesantemente limitata da questa divisione, poiché il recupero delle stesse è reso difficoltoso.
Infine, vi è il problema della manutenzione del sistema, difatti non vi è alcuna garanzia riguardo alla persistenza e alla sicurezza dei dati, ciò significa che non sono presenti sistemi che assicurino \textit{backup} dei dati e protezione contro accessi malintenzionati. \footnote{Ad eccezione della semplice password a protezione dell'account utente, comunque insufficiente}
\section{Il progetto SIMIOR}
Il Simior è un sistema informatico creato a inizio 2022 per sopperire alle limitazioni appena descritte, ed è attualmente utilizzato in fase sperimentale nel già citato reparto di terapia intensiva.
I destinatari di questo progetto sono i dottori, denominati utenti, i quali dopo aver fatto l'accesso tramite pagina web, potranno inserire le cartelle cliniche dei pazienti in cura, potendo tracciare l'andamento del ricovero e le statistiche del reparto.
\section{Differenze con l'implementazione attuale}
Nel SIMIOR viene fatto un uso di un sistema di gestione di basi dati relazionale (\textit{RDMBS, Relational Database Management System}) chiamato PostgreSQL, conosciuto e apprezzato sia in ambiente accademico che in realtà professionali affermate. I benefici dell'utilizzo di un RDMBS in questa situazione si riassumono in tre vantaggi principali:
\begin{itemize}
	\item Sopperire alle limitazioni di memoria di Access
	\item Costruire una struttura dati più efficiente e veloce nell'inserimento e nel recupero delle informazioni \footnote{in parte come beneficio ereditato dal tipo di tecnologia}
	\item Esprimere interrogazione più complesse
\end{itemize}
Questo ha permesso al progetto di organizzare al meglio i dati e semplificare l'organizzazione visiva degli stessi. 
Se con Excel l'utente viene sommerso di tutte le informazioni sul reparto e sui pazienti, in questo programma appare subito un quadro chiaro del sistema. Nella schermata principale sono presenti statistiche aggiornate automaticamente e la lista dei pazienti con le relative informazioni essenziali (es: data ricovero, codice paziente, motivo del ricovero). Solo una volta selezionato un paziente verranno mostrate più informazioni, sempre organizzate in relative sezioni, con eventuali grafici che meglio riescono a spiegare un determinato andamento.
Oltre a questi vantaggi, il SIMIOR è modellato sulle esigenze specifiche del reparto e viene adattato di conseguenza all'evolversi delle necessità, a differenza del binomio Excel-Access. \footnote{E' impossibile implementare funzionalità specifiche in questi software per via della loro natura \textit{closed-source} che non li rende liberamente modificabili}
La funzionalità richiesta che ha portato alla stesura di questa tesi è l'inserimento dei referti di laboratorio per poterne estrarre informazioni, in particolar modo gli antibiogrammi, grazie ad essa è possibile automatizzare il processo di trascrizione dei referti, semplificando e velocizzando il lavoro dell'utente, con conseguente riduzioni degli errori.


%\par\bigskip
%ref{img:nomeOggetto} -> per i riferimenti alle figure

%
%\begin{figure}[h!]
%	\centering
%	\includegraphics[width=.40\columnwidth]{images/meshExample}
%	\caption{\textit{Sezione di una mesh di superficie}}
%	\label{img:meshExample}
%\end{figure}
%

%\footnote{} -> per fare le citazioni in basso, va messo vicino al testo

%\ref{valore_etichetta} per riferimenti ad altre sezioni va usato in accoppiata con \label
