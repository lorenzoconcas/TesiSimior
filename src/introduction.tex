\chapter{Introduzione}
\section{Cenni storici}
Il \textit{Portable Document Format} è nato nel 1993 da Adobe Inc. con lo scopo di creare un formato file che permettesse di rappresentare documenti testuali o con immagini indipendentemente dall'hardware o dal software che usato per generare o visualizzare il documento stesso. Al momento la sua versione attuale è la 2.0 del 2017 ma lo standard più diffuso è la 1.4 di cui esiste la norma ISO 32000 stabilita nel 2008.
Essendo un formato libero, possono essere creati programmi che leggono e scrivono documenti PDF senza la necessità di pagare diritti di utilizzo del brevetto, tuttavia l'accesso ai documenti contenenti le specifiche ISO sono a pagamento.
Nonostante il successo di questa tecnologia, il PDF non è considerato come adeguato alla conservazione sul lungo termine perché la riproduzione può dipendere da sorgenti esterne quali Font o oggetti esterni (link, immagini allegate esternamente ecc).
\par\bigskip
\subsection{La struttura del PDF}
I PDF sono espressi tramite una sequenza di caratteri ASCII a 7 bit, pertanto le strutture di questo formato seguono tale codifica; fanno eccezione possibili oggetti binari che mantengono il loro flusso di bit originale senza alterazioni.
Ogni file inizia con un intestazione leggibile indicante la versione del formato:
\begin{center}
\begin{lstlisting}[language=C]
			%PDF-1.4	
\end{lstlisting}
\end{center}

La parte costituente e fondamentale di un PDF sono dei blocchi costruttivi chiamati "COS" \textit{"Carousel" Object Structure} (nome del progetto originale diventato in seguito Adobe Acrobat), questi blocchi nonostante il nome non sono dei veri e propri "oggetti" come nei linguaggi \textit{object-oriented}, ma rappresentano la struttura dei documenti.
Alcuni oggetti sono delimitati da dei cosidetti \texttt{marker} che possono essere parentesi tonde, angolari, particolari caratteri o parole chiave.
Esistono 9 tipi di oggetti \textit{COS}, ognuno con un compito specifico:
\begin{itemize}
	\item Oggetto \textit{null}: è indicato come la semplice sequenza di caratteri \texttt{"null"}, è comunemente utilizzato per indicare la mancanza di un valore
	\item Oggetti booleani: sono la rappresentazione diretta dei due valori dell'algebra booleana, sono indicati come \texttt{"true"} e \texttt{"false"}, alcuni PDF scritti scorrettamente presentano variazioni nelle maiuscole (es \texttt{TRUE, True, FALSE o False})
	\item Oggetti numerici : possono essere di due tipi, interi o reali, equivalenti ai rispettivi tipi matematici.\newline I tipi interi consistono in uno o più cifre decimali, precedute opzionalmente dal segno e rappresentano un valore in base 10. Di seguito alcuni esempi
	 \begin{itemize} 
	 	\item 1 
	 	\item -2134
	 	\item 96
	 \end{itemize}
	 I tipi reali presentano in più rispetto agli interi il punto come separazione fra parte intera e parte decimale, non è supportata la notazione esponenziale ne le radici non decimali. Alcuni esempi
	 \begin{itemize}
	 \item 0.0099
	 \item .200
	 \item -3.1415
	 \end{itemize}
	 \item Oggetti di tipo "Nome": Sono una sequenza di caratteri univoca in formato UTF-8 (escluso il carattere ASCII \texttt{null}) preceduta sempre da una barra obliqua (\texttt{/}). E' usato per definire un set di valori fissi.
	 \item Oggetti stringa: delle semplici serie di byte scritti come caratteri, racchiusi fra parentesi tonde o come dati in formato esadecimale racchiusi fra parentesi angolari. Una stringa letterale consiste in un arbitrario numero di caratteri 8 bit racchiusi fra parentesi. Dato che in questo caso qualsiasi carattere può apparire nella stringa le parentesi non bilanciate e il backslash la procedura di escape di questi caratteri viene fatta tramite un ulteriore backslash. Oltre a questo può essere usata la notazione \texttt{\textbackslash ddd}. Le stringhe letterali possono essere di vari tipi:
		\begin{itemize}
		\item ASCII : Semplice sequenza di bytes contenente soltanto caratteri ASCII
		\item PDFDocEncoded: Sequenza di byte codificata secondo lo standard ISO 32000-1:2008 
		\item Text: Sequenza codifica come PDFDocEncoding o come UTF-16BE (ossia con Byte Order Marker (BOM) in testa
		\item Date: Una stringa ASCII che segue le direttiva del formato ISO 32000-1:2008 formattata come segue : \texttt{D:YYYYMMDDHHmmSSOHH’mm}
	\end{itemize}
	 \item Oggetti array: Sono collezioni eterogenee di altri oggetti COS racchiuse fra parentesi quadre e separati da un \texttt{white-space}.
	 \item Oggetti dizionario: E' il tipo oggetto più comune nei PDF ed è la rappresentazione diretta delle collezioni \texttt{chiave-valore}, le chiavi sono sempre degli oggetti di tipo \texttt{nome}. I dizionari sono racchiusi fra doppie parentesi angolari (\texttt{<<>>}) e non vi è limite alla loro dimensione
	 \item Oggetti stream (flusso): Sono sequenze arbitrarie di byte potenzialmente illimitati, compressi o codificati. Vengono utilizzati per immagazzinare grandi blocchi di dati in altri formati standardizzati (per esempio \texttt{font, immagini, json}, ecc...\newline Gli stream sono sempre preceduti da un oggetto dizionario che ne descrive alcuni attributi fondamentali quali lunghezza del contenuto (obbligatoria) e tipo, questo dizionario è chiamato \texttt{stream dictionary}.\newline Gli stream sono delimitati dai marker \texttt{stream} e \texttt{endstream}
	\end{itemize}

















%
%\par\bigskip
%Nel corso degli ultimi decenni si è assistito a un costante miglioramento della computer grafica 3D la quale, è ora in grado di fornire uno strumento capace di rappresentare una realtà tanto verosimile all'attuale da rendere difficile la distinzione delle due. 
%Al contrario, la logica insita alla base della riproduzione di oggetti solidi è rimasta pressoché invariata, caratterizzandosi per il frequente impiego di mesh esclusivamente di superficie.
%Grazie a quest'ultima è possibile raffigurare un oggetto reale sfruttando una maglia costituita da numerose facce, dette poligoni, in grado di assumerne l'aspetto rivestendolo esternamente come fosse un involucro vuoto (Figura \ref{img:meshExample}). 
%
%\begin{figure}[h!]
%	\centering
%	\includegraphics[width=.40\columnwidth]{images/meshExample}
%	\caption{\textit{Sezione di una mesh di superficie}}
%	\label{img:meshExample}
%\end{figure}
%
%Allo sviluppo della potenza di calcolo degli elaboratori è proporzionalmente corrisposto il miglioramento della qualità delle mesh, essendo stati questi ultimi in grado di gestire modelli via via più complessi senza doverne mutare il funzionamento intrinseco. L'esigenza di rivoluzionare questo sistema è stata percepita esclusivamente all'interno di specifici ambienti di lavoro, in cui sono state introdotte delle mesh non più di superficie ma volumetriche. Un valido esempio di questa tendenza è ravvisabile in ambito medico: a seguito di una scansione, questa tecnologia ha reso possibile la ricostruzione virtuale del corpo del paziente, interamente o una sua porzione, rispettandone l'aspetto interno, oltre che quello esterno e permettendo l'identificazione di una più corretta diagnosi (Figura \ref{img:brainMesh}).
%
%\begin{figure}[h!]
%	\centering
%	\includegraphics[width=.40\columnwidth]{images/brainMesh}
%	\caption{\textit{Sezione di una mesh volumetrica \cite{brainMeshBib}}}
%	\label{img:brainMesh}
%\end{figure} 
%
%% COS’È UNA MESH VOLUMETRICA
%La mesh volumetrica dunque, non si limita esclusivamente a rappresentare lo strato più esteriore del modello, ma contiene anche informazioni relative a quello interno. La sua struttura non risulterà più formata da un insieme di poligoni, bensì costituita da un insieme discreto di \textit{voxel}\footnote{Il voxel adotta lo stesso comportamento di un qualsiasi pixel, ma su un piano tridimensionale. Visivamente assimilabile e definibile come un parallelepipedo, il voxel contiene al suo interno informazioni relative, ad esempio, alla sua posizione e al suo colore.} o poliedri che, uniti tra di loro, compongono il modello nella sua interezza. 
%
%% VANTAGGI 
%Nel campo della computer grafica, il suo uso è stato limitato alla simulazione di sostanze liquide o gassose ma potrebbe venire applicato anche per solidi. In virtù della sua particolare struttura infatti, questa tipologia di mesh può fornire all'utente una maggiore interazione con essa, consentendogli di manipolarla dinamicamente andando a simularne sezioni, esplosioni o divisioni.
%Ad averne particolare beneficio sarebbero il campo dei simulatori o dei videogame poiché le azioni sopraccitate, sono solitamente camuffate con l'uso di mesh precalcolate, producendo spesso effetti visibili sgradevoli e poco verosimili. 
%
%% PANORAMICA DEL MIO STUDIO 
%L'obiettivo posto da questo studio è di implementare un sistema di interazione delle mesh volumetriche all'interno del \textit{game engine} "Unity": i risultati raggiunti sono stati  concretizzati nella produzione di diverse demo che ne evidenziano l'efficacia applicativa. La realizzazione di questo progetto è stata correlata alla produzione di un'analisi prestazionale, finalizzata all'identificazione del limite massimo di poliedri processabili per ottenere un risultato soddisfacente: nell'esaminare le prestazioni dello script connesso a questi processi, infatti, si è evidenziato come all'aumentare della complessità dei modelli corrispondesse un loro maggiore tempo di caricamento\footnote{Per maggiori approfondimenti relativi a quest'aspetto si rimanda al capitolo \ref{risultati}.}. 