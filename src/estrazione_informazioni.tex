\chapter{Il PDF e l'Estrazione delle informazioni}
Prima di poter estrarre dati utili dai documenti PDF è necessario operare delle trasformazioni o delle rappresentazioni, dato che, per natura del PDF non vi sono dei dati utili al nostro scopo direttamente leggibili.

\section{Cosa sono i PDF?}
PDF è un acronimo che sta per \textit{Portable Document Format} è uno formato standard creato nel 1993 allo scopo di rappresentare documenti testuali o con immagini indipendentemente dall'hardware o dal software che usato per generare o visualizzare il documento stesso. I file costruiti in questo formato sono rappresentati tramite una sequenza di caratteri ASCII, all'interno possiamo individuare quattro strutture:

\begin{itemize}
	\item Header
	\item Body
	\item Cross-Reference Table
	\item Trailer
\end{itemize}
Ciò che viene presentato visivamente all'utente è contenuto nella sezione \texttt{body} del PDF, su questo flusso di caratteri si opera per ottenere le informazioni volute.
\newpage
\section{Tecniche di estrazione}
Le tecniche di estrazione sono principalmente due:
\begin{itemize}
	\item Conversione in formati testuali
	\item Lettura diretta del flusso \texttt{Body}
\end{itemize}

Nel primo caso si cerca di convertire (tramite librerie o servizi appositi) il documento in un file di testo dove poter effettuare l'estrazione. Questo metodo è stato scartato durante i test d'implementazione preliminari per via del risultato insoddisfacente, difatti i tentativi di conversione (con strumenti già esistenti e librerie dedicate) non riuscivano a convertire le tabelle in testo ma venivano generate delle immagini. 
Non è stato possibile utilizzare librerie dedicate all'estrazione di tabelle poiché gli antibiogrammi hanno una posizione fissa.
Nel secondo caso la realizzazione di un prodotto completo avrebbe richiesto studi molto più lunghi e fuori dallo scopo della tesi.
\section{La libreria utilizzata}
In un primo momento si è tentato di utilizzare la stessa libreria responsabile della generazione dei documenti, ma limitazioni di licenza e funzionalità hanno favorito la concorrente open-source Apache PDFBox che si è rivelata oltretutto più versatile e adatta allo scopo prefissato.
\section{L'approccio iniziale e le problematiche}
Il primo approccio si è basato sulla semplice estrazione del testo linea per linea dai referti con l'applicazione di specifiche regex. 
Avendo inizialmente soltanto un referto su cui effettuare i test non è apparso subito l'evidente problema sorto in seguito con gli antibiogrammi multi-microrganismo.
L'algoritmo di estrazione era dunque costituito dai seguenti passaggi:
\begin{enumerate}
\item Scorrere la lista fino a trovare la stringa corrispondente alla parola "Microrganismo 1" seguito dal nome del microrganismo
\item Scorrere di 2 posizioni (equivalenti a saltare la riga "Microrganismo 1" e "ANTIBIOTICO MIC")
\item Verificare ed estrarre le informazioni riguardo alla riga della tabelle tramite regex
\item Ripetere fino all'arrivo della riga contenente la legenda.
\end{enumerate}
Il problema di questo metodo è che nelle tabelle con più gruppi MIC-Sensibilità non si riesce ad associare con il gruppo al corretto microrganismo poiché è possibile che alcune celle siano vuote.
\section{Il secondo approccio}
Il secondo approccio, che apporta la correzione più significativa rivede completamente la logica di estrazione e aggiunge delle modifiche alla libreria PDFBox.
Il concetto fondamentale di quest'approccio è un estrazione "ibrida" ossia si ricerca l'inizio della tabella come nel metodo precedente, ma l'estrazione dei dati vera è propria viene effettuata calcolando la posizione delle varie celle.
\newline
Infatti anche se la posizione e la dimensione della tabella non è costante, le celle invece lo sono, permettendo un calcolo preciso della loro posizione e conseguente estrazione.
\subsection{Modifica della libreria}
Per ottenere la posizione degli elementi si è proceduto ad estendere la classe \textit{PDFTextStripper} della libreria aggiungendo prima una lista di oggetti di tipo \texttt{<string, List<TextPosition>}, ogni oggetto di questa lista è composto da una coppia chiave-valore in cui la chiave è una qualsiasi parola estratta e la chiave è una lista di posizioni in cui ogni lettera ha una sua coordinata.
Come seconda cosa è stato modificato il metodo \textit{writeString} in modo che inserisse nella lista prima citata queste nuove informazioni lasciando però inalterato il resto delle istruzioni.
\newline
\subsection{Localizzazione delle celle}
Una volta generate queste informazioni è possibile rappresentare l'algoritmo come segue:
\begin{enumerate}
	\item Si effettua una prima ricerca della tabella in modo analogo al metodo precedente
	\item Si continua a scorrere la lista e si tiene traccia del numero di microrganismi presenti nella tabella trovata
	\item Finita l'enumerazione dei microrganismi, si inizia a scorrere la nuova lista della classe modifica fino a trovare la parola chiave "ANTIBIOTICI", seguita da tante stringhe "MIC" quanti sono i microrganismi
	\item L'elemento puntato dalla lista sarà il nome dell'antibiotico, da qui si calcolano le dimensioni e le posizioni delle celle MIC e Sensibilità
	\item Definiamo tre "regioni", una per estrarre il nome dell'antibiotico alla linea selezionata, una per quella successiva e una per la legenda
	\item Si tenta la prima estrazione del testo, seguita poi, per ogni microrganismo dai seguenti passi:
	\begin{enumerate}
		\item Si calcola la cella MIC le cui coordinate saranno: distanza colonna antibiotico $ + 156 + $ (n° microrganismo)$*42$
		\item Discorso analogo per la cella Sensibilità che ha coordinate : distanza colonna MIC $ + $ (n° microrganismo) $*30*$
		\item Vengono definite altre due regioni, si tenta l'estrazione e si inseriscono le informazioni raccolte nell'antibiogramma
		\end{enumerate}
		\item Fatto questo, si scorre la lista delle parole fino a trovare o la legenda (che indica la fine della tabella e dell'estrazione) o il prossimo antibiotico (si confronta il testo con il dato estratto prima). Nel secondo caso si ferma lo scorrere della lista e si riparte del punto 4.
		\end{enumerate}
Questi 7 passaggi sono ripetuti per ogni pagina del PDF, ed è una soluzione quasi definitiva ma non esente da difetti. Cosa succede se ci sono più tabelle per pagina? Semplicemente viene rilevata soltanto la prima tabella e le successive vengono ignorate.
\section{La soluzione finale}









