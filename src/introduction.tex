\chapter{Introduzione}
Nelle Unità di Terapia Intensiva degli ospedali italiani viene effettuata, dal 2005, la sorveglianza delle Infezioni Correlate all'Assistenza (ICA) motivata dal fatto che i pazienti ricoverati in questi reparti, presentano un rischio maggiore (dalle 5 alle 10 volte) di contrarre un ICA, sia per fattori intrinseci che estrinseci, sia perché le UTI stesse sono epicentro di problemi emergenti di ICA. Oltre a questi due importanti motivi, la scelta di effettuare il monitoraggio è dovuta alla volontà di allinearsi ai progetti europei preesistenti e diventarne quindi componente della rete HAI-Net \footnote{Healthcare-associated Infections Surveillance Network},  
unendosi prima al network HELICS \footnote{Hospital in Europe Link for Infection Control through Surveillance} e successivamente al progetto IPSE \footnote{Improving Patient Safety in Europe}.
Da questo impegno nasce il progetto SPIN-UTI 
\footnote{Sorveglianza attiva Prospettica delle Infezioni Nosocomiali nelle Unità di Terapia Intensiva} il cui obbiettivo è quello di assicurare la standardizzazione delle definizioni, della raccolta dei dati e delle procedure di feedback da parte delle strutture ospedaliere partecipanti alla sorveglianza nazionale ed europea delle ICA nelle UTI, col fine ultimo di migliorare la qualità dell'assistenza a livello europeo nelle UTI stesse. \cite{progettoSpinuti}
\newpage
\section{Soggetto di studio}
Il reparto di terapia intensiva del Policlinico Ospedaliero Duilio Casula, al pari degli altri reparti nel sistema sanitario nazionale, adotta il sistema SPIN-UTI , con un 
implementazione informatica inadeguata per gli standard moderni. Per sopperire a queste insufficienze la facoltà di Informatica dell'Università di Cagliari ha creato un nuovo sistema informatico denominato SIMIOR\footnote{\textbf{S}istema \textbf{I}nformativo per il \textbf{M}onitoraggio delle \textbf{I}nfezioni \textbf{O}spedaliere nei \textbf{R}eparti di Rianimazione}, sviluppato su misura per le esigenze del reparto. I dati inseriti nel progetto contribuiscono al calcolo del punteggio di SPIN-UTI e le relative statistiche, ma la procedura di inserimento richiede attualmente un'operazione manuale da parte dei medici autorizzati. Ne conseguono, dunque, vari problemi, che spaziano dall'errore umano nella trascrizione alla pesantezza dell'operazione data dalla mole di dati da trascrivere.
In questa tesi verrà illustrato lo studio e i punti cardine del lavoro effettuato per aggiungere al SIMIOR la funzionalità di analisi ed estrazione delle informazioni dai referti. Nelle prossime pagine verrà mostrata la situazione attuale di SPIN-UTI nell'ospedale, la soluzione proposta e i suoi dettagli, in particolare modo come sono costituiti i referti PDF e le tecnologie utilizzate per creare gli automatismi necessari a ricavarne informazioni, i problemi riscontrati durante il processo e i passi compiuti per arrivare ai risultati ottenuti.
Oltretutto, il lavoro svolto per arrivare a questo risultato è stato utile per approfondire le conoscenze del funzionamento dei sistemi di distribuzioni web quali AWS\footnote{Amazon Web Services} che è l'insieme dei servizi sul quale il progetto viene eseguito e distribuito, tutte le tecnologie legate al cloud computing e quelle necessarie per la creazione di un software ad uso clinico. 


