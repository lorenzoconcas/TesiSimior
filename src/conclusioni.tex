\chapter{Conclusioni}
\section{Risultati}
I risultati sono stati raccolti da due fonti: le mail scambiate con i medici e i feedback rilasciati nella relativa sezione inserita nel progetto per questa tesi.
Il sistema è stato giudicato ottimo sia per la semplicità che per la correttezza dell'implementazione, con una volontà da parte degli utenti di sfruttare il più possibile il sistema e trovare una soluzione agli impedimenti burocratici incontrati nella fase di distribuzione del progetto (si tratta dei problemi legati alla privacy citati inizialmente e dei costi di gestione del progetto)
\section{Sviluppi futuri}
L'estrazione delle tabelle antibiogramma è soltanto una parte dell'automazione implementabile tramite questo sistema, è infatti possibile espandere le funzioni per ottenere dai referti qualsiasi tipo di informazione. 
Sono attualmente implementati, ma non utilizzati, metodi per estrarre informazioni in modo mirato, generando un livello di astrazione utile per usi futuri che esulano il programmatore dal riscrivere parti complesse di codice.
Un esempio sono le informazioni contenute nella sezione anagrafica (figura \ref{fig:header}) per cui il sistema è già predisposto ad estrarne i dati, i quali permetterebbe la creazione di schede ricovero da un semplice referto.

