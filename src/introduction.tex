\chapter{Introduzione}
\section{SPIN-UTI}
L'acronimo SPIN-UTI (Simplified Prognostic INSevere Sepsis in ICU) indica un sistema di punteggio pensato per valutare il rischio di mortalità dei pazienti affetti da sepsi grave ricoverati nei reparti di terapia intensiva (UTI).
\newline
Questo sistema di punteggio, in uso nel reparto di terapia intensiva del Presidio Ospedaliero Duilio Casula è implementato a livello informatico con l'utilizzo dei software Microsoft Excel e Access, il primo utilizzato per inserire le informazioni e per effettuare i calcoli tramite formule, il secondo utilizzato come base dati.
Viene logico intuire l'inidoneità del sistema attuale, soprattutto sul fronte organizzativo dei dati.

\section{Il progetto SIMIOR}
Il Simior è un sistema informatico creato a inizio 2022 per sopperire alle limitazioni dell'attuale implementazione del sistema di punteggio SPIN-UTI attualmente utilizzato nel già citato reparto di terapia intensiva.
I destinatari di questo progetto sono i dottori, denominati utenti, che accedendo tramite pagina web potranno inserire le cartelle cliniche dei pazienti in cura, potendo tracciare l'andamento del ricovero e le statistiche del reparto.
\section{Differenze con l'implementazione attuale}
La principale limitazione dell'implementazione attuale di SPIN-UTI è il limite di informazioni inseribili per ogni scheda-paziente, con la conseguente duplicazione delle stesse al fine di memorizzare tutte le informazioni sulla la degenza. 

Ne risulta un sistema poco ordinato e maggiormente soggetto a errori durante la copia delle informazioni essenziali.
\newline
Il simior è modellato sulle esigenze specifiche del reparto e viene adattato di conseguenza all'evolversi delle necessità.
La necessità che ha portato alla stesura di questa tesi è l'estrazione delle informazioni dai referti di laboratori, nello specifico gli antibiogrammi.
Questa funzionalità permette, in maniera semplice per l'operatore, di allegare un referto in formato PDF ed ottenere in una tabella, visivamente analoga ad altre parti del progetto, i dati contenuti nel documento originale.
\section{Tecnologie utilizzate}
SIMIOR è in sostanza un sito web realizzato con le tecnologie standard del settore, in gran parte viste durante il corso di studi e possono essere riassunte in tre parti
\begin{itemize}
	\item Back-End, scritto Java 8 ed eseguito sul server open-source GlassFish
	\item Front-End, basato sulle JavaServerPages (JSP) con l'aggiunta del framework Bootstrap e della libreria JavaScript jQuery
	\item Database, gestito dal DBMS PostgreSQL
\end{itemize}
Per poter sviluppare il progetto sono stati utilizzati vari software, quali l'IDE IntelliJ IDEA e PgAdmin per la gestione del database,
mentre la distribuzione del prodotto finale è affidata ai servizi AWS di Amazon, sui cui è presente un istanza EC2 per Simior e un'istanza RDS (\textit{Relational Database Service}) dedicata ad ospitare il DBMS e i dati.
Al fine di permettere agli utenti di testare il software senza modificare la cosiddetta "produzione" (ossia la versione stabile destinata al pubblico finale) è stata creata un istanza privata ospitata nei server della facoltà in cui sono stati implementati manualmente servizi analoghi a quelli di Amazon.
Questo percorso di tesi ha quindi avuto l'ulteriore utilità di far conoscere e comprendere il funzionamento di sistemi di distribuzione a livello professionale, sperimentando in prima persona l'organizzazione del software e gli strumenti utilizzati in ambito reale.



%comandi
%\par\bigskip
%ref{img:nomeOggetto} -> per i riferimenti alle figure

%
%\begin{figure}[h!]
%	\centering
%	\includegraphics[width=.40\columnwidth]{images/meshExample}
%	\caption{\textit{Sezione di una mesh di superficie}}
%	\label{img:meshExample}
%\end{figure}
%

%\footnote{} -> per fare le citazioni in basso, va messo vicino al testo

%\ref{valore_etichetta} per riferimenti ad altre sezioni va usato in accoppiata con \label
