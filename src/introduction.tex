\chapter{Introduzione}
\section{Il progetto SIMIOR}
Il Simior è un sistema informatico creato a inizio 2022 per sopperire alle limitazioni attuali del sistema SPIN-UTI attualmente utilizzato dal reparto di terapia intensiva del Presidio Ospedaliero Duilio Casula.
La principale limitazione di SPIN-UTI è il limite di informazioni inseribili per ogni scheda-paziente, con la conseguente duplicazione delle stesse al fine di memorizzare tutte le informazioni sulla la degenza. Ne risulta un sistema poco ordinato e maggiormente soggetto a errore durante la copia delle informazioni essenziali.
Il simior è modellato sulle esigenze specifiche del reparto e viene adattato di conseguenza all'evolversi delle necessità.
La necessità che ha portato alla stesura di questa tesi è l'estrazione delle informazioni dai referti di laboratori, nello specifico gli antibiogrammi.
Questa funzionalità permette, in maniera semplice per l'operatore, di allegare un referto in formato PDF ed ottenere in una tabella, visivamente analoga ad altre parti del progetto, i dati contenuti nel documento originale.
\section{Differenze con SPIN-UTI}
\section{Dettagli sul SIMIOR}
Il Simior è costituito da un applicazione web scritta il cui backend è scritto in Java e il front-end 
%comandi
%\par\bigskip
%ref{img:nomeOggetto} -> per i riferimenti alle figure

%
%\begin{figure}[h!]
%	\centering
%	\includegraphics[width=.40\columnwidth]{images/meshExample}
%	\caption{\textit{Sezione di una mesh di superficie}}
%	\label{img:meshExample}
%\end{figure}
%

%\footnote{} -> per fare le citazioni in basso, va messo vicino al testo

%\ref{valore_etichetta} per riferimenti ad altre sezioni va usato in accoppiata con \label
