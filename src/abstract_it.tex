\vspace{4cm}
Nei reparti di terapia intensiva italiani (UTI) è in vigore un sistema di punteggio utilizzato per valutare il rischio di mortalità nei pazienti affetti da sepsi grave. Non esiste attualmente un sistema informatico standardizzato per il calcolo di questo punteggio, di conseguenza il Policlinico Ospedaliero Duilio Casula in collaborazione con la Facoltà di Scienze dell'Università di Cagliari hanno collaborato per creare il progetto SIMIOR. Per calcolare questo punteggio ed estrapolare statistiche sono però necessari dati delle cartelle cliniche, che vengono inseriti manualmente.
\par\bigskip
Questa tesi analizza l'implementazione dei necessari automatismi per accelerare il processo di inserimento dei referti medici nel sistema, la struttura dei referti medici e le modifiche attuate al sistema SIMIOR.